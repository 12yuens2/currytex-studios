\documentclass[11pt]{article}
\usepackage[utf8]{inputenc}
\usepackage[T1]{fontenc}
\usepackage{fixltx2e}
\usepackage{graphicx}
\usepackage{longtable}
\usepackage{float}
\usepackage{wrapfig}
\usepackage{rotating}
\usepackage[normalem]{ulem}
\usepackage{amsmath}
\usepackage{textcomp}
\usepackage{marvosym}
\usepackage{wasysym}
\usepackage{amssymb}
\usepackage[hidelinks]{hyperref}
\usepackage{listings}
\usepackage{xcolor}
\usepackage{amsthm}



\newcommand{\n}[0]{\\[\baselineskip]}


\author{140011146}
\title{CS4303 Practical 3}

\begin{document}

\maketitle



\section{Title}

\section{Genre}
Management simulator.

\section{Player}
The player plays as a manager of a tech start-up. The player can hire workers and manage what they do.


\section{Rules and mechanics}
The idea of the game is to build the start-up company to be bigger, by gaining reputation and therefore earning more money and growing the company. The core gameplay mechanic is managing how the workers work by dragging their avatar boxes to different locations to start new activities. Some activities add progress to the company, such as getting and working on new projects to gain money/reputation. Other activities include improving the workers individually, such as giving them stats. 
\n
\subsection{Workers}

\section{Opponent and goal}
The opponent of the game is a combination of time limit, milestone goals  that have to be reached and regular time intervals. This represents higher up ``bosses" who demands good performance from the player and the need to pay salary to the workers. 
\n
The opponent is also the goal that has to be met. The milestone goals are in place to make sure the player is progressing while the final goal is how the player wins the game. 


\section{Context}

\section{Design}

\subsection{Testing}

\subsection{Further work}

\section{Platform}

\section{Evaluation}

\end{document}

